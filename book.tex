%\documentclass[11pt]{scrbook} % use larger type; default would be 10pt

% WITH BLEED
% US Trade => 6x9, with a 0.125 bleed
% Adjust images size and gutter so tabs bleed by .125
% See https://www.createspace.com/Products/Book/InteriorPDF.jsp
\documentclass[paper=6.14in:9.21in,pagesize=pdftex,11pt,twoside,openright]{scrbook}
%openright
% Paper width
% W = 6.125in (6+0.125 --- bleed)
% Paper height
% H = 9.25in (9+2*.125 --- bleed)
% Paper gutter
% BCOR = 0.375in (0.5+0.5-0.625 --- margin with bleed)
% Margin (0.5in imposed on lulu, recommended on createspace)
% m = 0.625in (0.5+0.125 --- bleed)
% Text height
% h = H - 2m = 8in
% Text width
% w = W - 2m - BCOR = 4.5in
\areaset[0.375in]{4.5in}{8in}
\usepackage[utf8]{inputenc} % set input encoding (not needed with XeLaTeX)

%%% Examples of Article customizations
% These packages are optional, depending whether you want the features they provide.
% See the LaTeX Companion or other references for full information.

%\usepackage{mitpress}
\usepackage{framed}

\usepackage[top=1in, bottom=1in, left=1in, right=1in]{geometry}

\usepackage{graphicx} % support the \includegraphics command and options
%\usepackage{wrapfig}

% \usepackage[parfill]{parskip} % Activate to begin paragraphs with an empty line rather than an indent

%%% PACKAGES
\usepackage{booktabs} % for much better looking tables
\usepackage{array} % for better arrays (eg matrices) in maths
\usepackage{paralist} % very flexible & customisable lists (eg. enumerate/itemize, etc.)
\usepackage{verbatim} % adds environment for commenting out blocks of text & for better verbatim
\usepackage{subfig} % make it possible to include more than one captioned figure/table in a single float
% These packages are all incorporated in the memoir class to one degree or another...
\usepackage{amsmath}
\usepackage{amsfonts}
\usepackage[font={small,it}]{caption}
\usepackage{url}
\usepackage{hyperref}
\usepackage{harvard}
\usepackage{marginnote}

\usepackage{makeidx}
\usepackage{idxlayout}

%%% HEADERS & FOOTERS
%\usepackage{fancyhdr} % This should be set AFTER setting up the page geometry
%\pagestyle{fancy} % options: empty , plain , fancy
%\renewcommand{\headrulewidth}{0pt} % customise the layout...
%\lhead{}\chead{}\rhead{}
%\lfoot{}\cfoot{\thepage}\rfoot{}

%%% SECTION TITLE APPEARANCE
\usepackage{sectsty}
\allsectionsfont{\sffamily\mdseries\upshape} % (See the fntguide.pdf for font help)
% (This matches ConTeXt defaults)

%%% ToC (table of contents) APPEARANCE
\usepackage[nottoc,notlof,notlot]{tocbibind} % Put the bibliography in the ToC
\usepackage[titles,subfigure]{tocloft} % Alter the style of the Table of Contents
\renewcommand{\cftsecfont}{\rmfamily\mdseries\upshape}
\renewcommand{\cftsecpagefont}{\rmfamily\mdseries\upshape} % No bold!

\newcommand{\screencast}[2]{
  \marginnote{\href{#1}{\includegraphics[width=1cm]{figs/youtube/#2}}}}


%%% END Article customizations

%%% The ``real'' document content comes below...
\makeindex

%\title{Introduction to Autonomous Robots}
%\author{Nikolaus Correll}
%\date{} % Activate to display a given date or no date (if empty),
         % otherwise the current date is printed 

\begin{document}
%\maketitle

\thispagestyle{empty}
\begin{flushleft}
Nikolaus Correll\\
Introduction to Autonomous Robots, v1.7, \today\\
Magellan Scientific\\
%ISBN-13: 978-1493773077
ISBN-13: 978-0692700877
\end{flushleft}

\vfill

\begin{figure}[!h]
\includegraphics[width=1in]{figs/by-nc-nd}
\end{figure}

This book is licensed under a Creative Commons Attribution-NonCommercial-NoDerivs 3.0 Unported License. You are free to share, i.e., copy, distribute and transmit the work under the following conditions: You must attribute the work to its main author, you may not use this work for commercial purposes, and you may not alter, transform, or create derivatives of this work, except for written permission by the author. For more information, please consult \url{http://creativecommons.org/licenses/by-nc-nd/3.0/deed.en_US}.


\cleardoublepage
\thispagestyle{empty}
\topskip0pt
\vspace*{\fill}
\begin{center}
For Arthur, Tatiana, Benedict and Silvester\\
future robot users
\end{center}
\vspace*{\fill}

\tableofcontents

\chapter*{Preface}
This book provides an algorithmic perspective to autonomous robotics to students with a sophomore-level of linear algebra and probability theory. Robotics is an emerging field at the intersection of mechanical and electrical engineering with computer science. With computers becoming more powerful, making robots smart is getting more and more into the focus of attention and robotics research most challenging frontier. While there are a large number of textbooks on the mechanics and dynamics of robots that address sophomore-level undergraduates available, books that provide a broad algorithmic perspective are mostly limited to the graduate level. This book has therefore been developed not to create ``yet another textbook, but better than the others'', but to allow me to teach robotics to the 3rd and 4th year undergraduates at the Department of Computer Science at the University of Colorado. 

Although falling under the umbrella of ``Artificial Intelligence'', standard AI techniques are not sufficient to tackle problems that involve uncertainty, such as a robot's interaction in the real world. This book uses simple trigonometry to develop the kinematic equations of simple manipulators and mobile robots, then introduces path planning, sensing, and hence uncertainty. The robot localization problem is introduced by formally introducing error propagation, which leads to Markov localization, the Particle filter and finally the Extended Kalman Filter, and Simultaneous Localization and Mapping. 

Instead of focusing on the state-of-the-art solutions to a particular sub-problem, emphasis of the book is on a concise step-by-step development and recurrent examples that capture the essence of a problem, but might not necessarily be the best solution. For example, odometry and line-fitting are used to explain forward kinematics and least-squares solutions, respectively, and later serve as motivating examples for error propagation and the Kalman filter in a localization context.

Also, the book is explicitely robot-agnostic, reflecting the timeliness of fundamental concepts. Instead, a series of possible project-based curricula are described in an Appendix and available online, ranging from a maze-solving competition that can be realized with most miniature differential-wheel robots that include a camera to manipulation experiments with the Baxter robot, all of which can be entirely conducted in simulation. 

This book is released under a Creative Commons license, which allows anyone to copy and share this book, although not for commercial purposes and not to create derivatives of these works. This license comes very close to the ``copyright'' of a standard textbook, except that you are free to copy it for non-commercial purposes. I have chosen this format as it seems to maintain the best trade-off between a freely available textbook resource that others hopefully contribute to and maintaining a consistent curriculum that others can refer to. 

Writing this book would not have been possible without the excellent work of others before me, most notably ``Introduction to Robotics: Mechanics and Control'' by John Craig and ``Introduction to Autonomous Mobile Robots'' by Roland Siegwart, Illah Nourbakhsh and David Scaramuzza, and innumerable other books and websites from which I learned and borrowed examples and notation.

\begin{flushright}
Nikolaus Correll\\
Boulder, Colorado, \today
\end{flushright}

\input{chapters/introduction}
\input{chapters/locomotion}
\input{chapters/forwardkinematics1}
\input{chapters/inversekinematics}
\input{chapters/forwardkinematics3}
\chapter{Path Planning}\label{chap:pathplanning}
Path-planning is an important primitive for autonomous mobile robots that lets robots find the shortest---or otherwise optimal---path between two points. Optimal paths could be paths that minimize the amount of turning, the amount of braking or whatever a specific application requires. Algorithms to find a shortest path are important not only in robotics, but also in network routing, video games and understanding protein folding.

Path-planning requires a map of the environment and the robot to be aware of its location with respect to the map. We will assume for now that the robot is able to localize itself,  is equipped with a map, and capable of avoiding temporary obstacles on its way. How to create a map, how to localize a robot, and how to deal with uncertain position information will be major foci of the reminder of this  book. The goals of this chapter are to

\begin{itemize}
\item introduce suitable map representations,
\item explain basic path-planning algorithms  ranging from Dijkstra, to A*, D* and RRT,
\item introduce variations of the path-planning problem, such as coverage path planning.
\end{itemize}

\section{Map representations}\label{sec:maps}
In order to plan a path, we somehow need to represent the environment in the computer. We differentiate between two complementary approaches: discrete and continuous approximations. In a discrete approximation, a map is sub-divided into chunks of equal (e.g., a grid or hexagonal map) or differing sizes (e.g., rooms in a building). The latter maps are also known as \emph{topological maps}.\index{Topological map} Discrete maps lend themselves well to a graph representation. Here, every chunk of the map corresponds to a vertex (also known as ``node''), which are connected by edges, if a robot can navigate from one vertex to the other. For example a road-map is a topological map, with intersections as vertices and roads as edges, labeled with their length (Figure \ref{fig:pathproblem}). Computationally, a graph might be stored as an adjacency or incidence list/matrix. A continuous approximation requires the definition of inner (obstacles) and outer boundaries, typically in the form of a polygon, whereas paths can be encoded as sequences of points defined by real numbers. Despite the memory advantages of a continuous representation, discrete maps are the dominant representation in robotics.

For mapping obstacles, the most common map is the \emph{occupancy grid} map\index{Occupancy grid}. In a grid map, the environment is discretized into squares of arbitrary resolution, e.g. 1cm x 1cm, on which obstacles are marked. In a probabilistic occupancy grid, grid cells can also be marked with the probability that they contain an obstacle. This is particularly important when the position of the robot that senses an obstacle is uncertain. Disadvantages of grid maps are their large memory requirements as well as computational time to traverse data structures with large numbers of vertices. A solution to this is storing the grid map as \emph{k-d tree}.\index{k-d tree (data structure)} A k-d tree recursively breaks the environment into $k$ pieces. For $k=4$, an area is broken into four pieces. Each of these pieces is again broken into four pieces and so on, until the desired resolution is reached. These pieces can easily be stored in a graph with each vertex having four children, which are the four pieces the vertex is broken into, or is a leaf of the tree. What makes this data structure attractive is that not all vertices need to be broken down to the smallest possible resolution. Instead only areas, which contain obstacles need to be further broken down. A grid map containing obstacles and the corresponding k-d tree, here a quadtree, are shown in Figure \ref{fig:gridvskdtree}. There is no silver bullet, and each application might require a different solution that could be a combination of different map types.

\begin{figure}
\centering
\includegraphics[width=\textwidth]{figs/gridvskdtree.png}
\caption{A grid map and its corresponding quadtree (k-d tree).\label{fig:gridvskdtree}}
\end{figure}


There exist also every possible combination of discrete and continuous representation. For example, roadmaps for GPS systems are stored as topological maps that store the GPS coordinates of every vertex, but might also contain overlays of aerial and street photography or even 3D point clouds stored in a 8-d tree, also known as a \emph{Octree}\index{Octree}. These different maps are then used at different stages of the path planning stage. 


\section{Path-Planning Algorithms}
The problem to find a ``shortest'' path from one vertex to another through a connected graph is of interest in multiple domains, most prominently in the internet, where it is used to find an optimal route for a data packet. The term ``shortest'' refers here to the minimum cumulative edge cost, which could be physical distance (in a robotic application), delay (in a networking application) or any other metric that is important for a specific application. An example graph with arbitrary edge-lengths is shown in Figure \ref{fig:pathproblem}. 

\begin{figure}[!htb]
\centering
\includegraphics[width=0.6\textwidth]{figs/dijkstra}
\caption{A generic path planning problem from vertex I to vertex VI. The shortest path is I-II-III-V-VI with length 13. \label{fig:pathproblem}}
\end{figure}

\subsection{Robot embodiment}
In order to deal with the physical embodiment of the robot, which complicates the path-planning process, the robot is reduced to a point-mass and all the obstacles in the environment are grown by half of the longest extension of the robot from its center. This representation is known as configuration space \index{Configuration space}as it reduces the representation of the robot to its $x$ and $y$ coordinates in the plane. An example is shown in Figure \ref{fig:cspace}. The configuration space can now either be used as a basis for a grid map or a continuous representation. 

\begin{figure}[!htb]
\centering
\includegraphics[width=0.9\textwidth]{figs/configurationspace}
\caption{A map with obstacles and its representation in configuration space, which can be obtained by growing each obstacle by the robot's extension. \label{fig:cspace}}
\end{figure}


\subsection{Dijkstra's algorithm}\index{Dijkstra's Shortest Path Algorithm}
One of the earliest and simplest algorithms is Dijkstra's algorithm \cite{dijkstra1959note}. Starting from the initial vertex where the path should start, the algorithm marks all direct neighbors of the initial vertex with the cost to get there. It then proceeds from the vertex with the lowest cost to all of its adjacent vertices and marks them with the cost to get to them via itself if this cost is lower. Once all neighbors of a vertex have been checked, the algorithm proceeds to the vertex with the next lowest cost. Once the algorithm reaches the goal vertex, it terminates and the robot can follow the edges pointing towards the lowest edge cost.

In Figure \ref{fig:pathproblem}, Dijkstra would first mark nodes II, III and IV with cost 3, 5 and 7 respectively. It would then continue to explore all edges of node II, which so far has the lowest cost. This would lead to the discovery that node III can actually be reached in $3+1<5$ steps, and node III would be relabeled with cost 4. In order to completely evaluate node II, Dijkstra needs to evaluate the remaining edge before moving on and label node VI with $3+12=15$.

The node with the lowest cost is now node III (4). We can now relabel node VI with 14, which is smaller than 15, and label node V with $4+5=9$, whereas node IV remains at $4+3=7$. Although we have already found two paths to the goal, one of which better than the other, we cannot stop as there still exist nodes with unexplored edges and overall cost lower than 14. Indeed, continuing to explore from node V leads to a shortest path I-II-III-V-VI of cost 13, with no remaining nodes to explore.

As Dijkstra would not stop until there is no node with lower cost than the current cost to the goal, we can be sure that a shortest path will be found if it exists. We can say that the algorithm is \emph{complete}.\index{Complete (algorithm)}

As Dijkstra will always explore nodes with the least overall cost first, the environment is explored comparably to a wave front originating from the start vertex, eventually arriving at the goal. This is of course highly inefficient in particular if Dijkstra is exploring nodes away from the goal. This can be visualized by adding a couple of nodes to the left of node I in Figure \ref{fig:pathproblem}. Dijkstra will explore all of these nodes until their cost exceeds the lowest found for the goal. This can also be seen when observing Dijkstra's algorithm on a grid, as shown in Figure \ref{fig:dijkstragrid}. 

\begin{figure}[htb]
\centering
\includegraphics[width=0.9\textwidth]{figs/dijkstragrid}
\caption{Dijkstra's algorithm finding a shortest path from `S' to 'G' assuming the robot can only travel laterally (not diagonally) with cost one per grid cell. Note the few number of cells that remain unexplored once the shortest path (grey) is found, as Dijkstra would always consider a cell with the lowest path cost first.\label{fig:dijkstragrid}}
\end{figure}

\subsection{A*}\label{sec:astar}\index{A* Shortest Path Algorithm}
Instead of exploring in all directions, knowledge of an approximate direction of the goal could help avoiding exploring nodes that are obviously wrong to a human observer. Such special knowledge that such an observer has can be encoded using a \emph{heuristic function}\index{Heuristic function}, a fancier word for a ``rule of thumb''. For example, we could give priority to nodes that have a lower estimated distance to the goal than others. For this, we would mark every node not only with the actual distance that it took us to get there (as in Dijkstra's algorithm), but also with the estimated cost ``as the crows flies'', for example by calculating the Euclidean distance or the \emph{Manhattan distance}\index{Manhattan distance} between the vertex we are looking at and the goal vertex. This algorithm is known as A* \cite{hart1968formal}. Depending on the environment, A* might accomplish search much faster than Dijkstra's algorithm, and performs the same in the worst case. This is illustrated in Figure \ref{fig:astargrid} using the Manhattan distance metric, which does not allow for diagonal movements. 

\begin{figure}[htb]
\centering
\includegraphics[width=0.9\textwidth]{figs/astargrid}
\caption{Finding a shortest path from `S' to 'G' assuming the robot can only travel laterally (not diagonally) with cost one per grid cell using the A* algorithm. Much like Dijkstra, A* evaluates only the cell with the lowest cost, but takes an estimate of the remaining distance into account.\label{fig:astargrid}}
\end{figure}

An extension of A* that addresses the problem of expensive re-planning when obstacles appear in the path of the robot, is known as D* \cite{stentz1994optimal}. Unlike A*, D* starts from the goal vertex and has the ability to change the costs of parts of the path that include an obstacle. This allows D* to re-plan around an obstacle while maintaining most of the already calculated path.\index{D*}

A* and D* become computationally expensive when either the search space is large, e.g., due to a fine-grain resolution required for the task, or the dimensions of the search problem are high, e.g. when planning for an arm with multiple degrees of freedom. Solutions to these problems are provided by sampling-based path planning algorithms that are described further below.




\section{Sampling-based Path Planning}
The previous sections have introduced a series of complete algorithms for the path planning problem, i.e. they will find a solution eventually if it exists. Complete solutions are often unfeasible, however, when the possible state space is large. This is the case for robots with multiple degrees of freedom such as arms. In practice, most algorithms are only \emph{resolution complete}\index{Resolution complete}, i.e., only complete if the resolution is fine-grained enough, as the state-space needs to be somewhat discretized for them to operate (e.g., into a grid) and some solutions might be missed as a function of the resolution of the discretization.

Instead of evaluating all possible solutions or using a non-complete Jacobian-based inverse kinematic solution, sampling-based planners create possible paths by randomly adding points to a tree until some solution is found or time expires. As the probability to find a path approaches one when time goes to infinity, sampling-based path planners are \emph{probabilistic complete}\index{Probabilistic complete}. Prominent examples of sampling-based planners are \emph{Rapidly-exploring Random Trees} (RRT)\index{Rapidly-exploring Random Tree}\index{RRT}\cite{lavalle1998rapidly} and \emph{Probabilistic Roadmaps}\index{Probabilistic Roadmaps}\index{PRM}(PRM) \cite{kavraki1996probabilistic}. An example execution of RRT for an unknown goal, thereby reducing the path-planning problem to a search problem is shown in Figure \ref{fig:rrt}.

\begin{figure}
\centering
\includegraphics[width=\textwidth]{figs/irrt}
\caption{Counterclockwise: Random exploration of a 2D search space by randomly sampling points and connecting them to the graph until a feasible path between start and goal is found.\label{fig:rrt}}
\end{figure}

This example illustrates well how a sampling-based planner can quickly explore a large portion of space and refines a solution as time goes on. Whereas RRT can be understood as growing a single tree from a robot's starting point until one of its branches hit a goal, probabilistic road-maps create a tree by randomly sampling points in the state-space, testing whether they are collision-free, connecting them with neighboring points using paths that reflect the kinematics of a robot, and then using classical graph shortest path algorithms to find shortest paths on the resulting structure. The advantage of this approach is clearly that such a probabilistic roadmap has to be created only once (assuming the environment is not changing) and can then be used for multiple queries. PRMs are therefore a \emph{multi-query} path-planning algorithm.\index{Multi-query (path-planning)} In contrast, RRT's are known as \emph{single-query}\index{Single-query (path-planning)} path-planning algorithms.

In practice, the boundary between the different historic algorithms has become very diffuse, and single-query and multi-query variants of both RRT and PRM exist. It is important to note that there is no silver bullet algorithm/heuristic and even their parameter-sets are highly problem-specific. We will therefore limit our discussion on useful heuristics that are common to sampling-based planners.

\subsection{Basic Algorithm}
Let $ \mathcal{X}$ be a $ d$-dimensional state-space. This can either be the robot's state given in terms of translation and rotations (6 dimensions), a subset thereof, or the joint space with one dimension per possible joint angle. Let $ \mathcal{G} \subset \mathcal{X}$ be a  d-ball (d-dimensional sphere) in the state-space that is considered to be the goal, and $ t$ the allowed time. A tree planner proceeds as follows:

\begin{verbatim}
Tree=Init(X,start);
WHILE ElapsedTime() < t AND NoGoalFound(G) DO
 newpoint = StateToExpandFrom(Tree);
 newsegment = CreatePathToTree(newpoint);
 IF ChooseToAdd(newsegment) THEN
  Tree=Insert(Tree,newsegment);
 ENDIF
ENDWHILE
return Tree
\end{verbatim}

This process can be repeated on the resulting tree as long as time allows. This is known as an \index{AnyTime algorithm} \emph{AnyTime} algorithm. Given a suitable distance metric, the cost-to-goal can be stored at each node of the tree (much easier if growing the tree from the goal to start), which allows retrieving the shortest path easily. There are four key points in this algorithm:

\begin{enumerate}
\item Finding the next point to add to the tree (or discard) (StateToExpandFrom).
\item Finding out where and how to connect this point to the tree taking into account the robot kinematics (CreatePathToTree).
\item Testing whether this path is suitable, i.e., collision-free.
\item Finding the next point to add.
\end{enumerate}

A prominent method is to pick a random point in the state-space and connect it to the closest existing point in the tree or to the goal. This requires searching all nodes in the tree and calculating their distance to the candidate point. Other approaches put preferences on nodes with fewer out-degrees (those which do not yet have very many connections) and chose a new point within its vicinity. Both approaches make it likely to quickly explore the entire state-space.

If there are constraints imposed on the robot's path, for example the robot needs to hold a cup and therefore is not supposed to rotate its wrist, this dimension can simply be taken out of the state-space.

Once a possible path is found, this space can be reduced to the ellipsoid that bounds the maximal path-length. This ellipsoid can be constructed by mounting a wire of the maximum path length between start and goal and pushing it outward with a pen. All the area that can be reached with the pen constrained by the wire can contain a point that can possibly lead to a shorter path. This approach is particularly effective when running multiple copies of the same planner in parallel and exchanging the shortest paths once they are found \cite{otte2012}.

\subsection{Connecting Points to the Tree}
A new point is classically connected to the closest point already in the tree or to the goal. This can be done by calculating the distance to all points already in the tree. This does not necessarily generate the shortest path, however. A recent improvement has been made by RRT*, which connects the point to the tree in a way that minimizes the overall path length. This can be done by considering all points in the tree within a d-ball (on a 2D map, $d=2$, i.e. a circle) from of fixed radius from the point to add and finding the point that minimizes the overall path length to the start.

Adding a point to the tree is also a good time to take into account the specific kinematics of a robot, for example a car. Here, a local planner can be used to generate a suitable trajectory that takes into account the orientation of the vehicle at each point in the tree.

\subsection{Collision Checking}
Efficient algorithms for testing collisions deserve a dedicated lecture. While the problem is intuitive in configuration-space planning in 2D (the robot reduces to a point) and can be solved using a simple point-in-polygon test, the problem is more involved for manipulators that are subject to self-collision.

As collision checking takes up to 90\% of the execution time in the path planning problem, a successful method to increase computational speed is ``lazy collision evaluation''\index{Lazy collision avoidance}. Instead of checking every point for a possible collision, the algorithm first finds a suitable path. Only then, it checks every segment of this path for collisions. In case some segments are in collision, they are deleted and the algorithm goes on, but keeps the segments of the successful path that were collision-free.

\section{Path Smoothing}
As paths are randomly sampled, they will be most likely shakey and not optimal. For exampe, a grid-map will generate a series of sharp turns and a sampling-based approach will return zig-zag random paths. Results can be drastically improved by running an additional algorithm that smoothes the path. One way of doing this is to connect points of the path using splines, curves or even trajectory snippets that are known to be feasible for a specific platform. Alternatively, one can also use a model of the actual platform and use a feedback controller such as described in Section \ref{sec:fbmobile} for mobile robots and Section \ref{sec:invjac} for arms, sample a series of points in front of the robot, and generate a trajectory that the robot can actually drive. When combined with dynamics, this approach is known as \emph{model-predictive control}\index{Model-predictive control}. Care needs to be taken, however, that the resulting paths are indeed collision free. 

\section{Planning at different length-scales}
In practice, no one map representation and planning algorithm might be sufficient. To plan a route for a car, for example, might involve a coarse search over the street network such as performed by your car's navigation system, but not involve planning which lane to actually choose. Planning lanes and how to navigate round-abouts and intersections will then involve another layer of discrete planning. How to actually move the robot within a lane and avoid local obstacles, might then be best done with a sampling-based planning algorithm. Finally, trajectories need to be turned into wheel speeds and turn angles, possibly using some form of feedback control. This hierarchy is depicted in Figure \ref{fig:planninglayers}. Here, downward-pointing arrays indicate input that one planning layer provides to the one below. Upward-pointing arrows instead indicate exceptions that cannot be handled by the lower levels. For example, a feedback controller cannot handle obstacles well, requiring the sampling-based planning layer to come up with a new trajectory. Should the entire road be blocked, this planner would need to hand-off control the lane-based planner. A similar case can be made for manipulating robots, which also need to combine multiple different representations and controllers to plan and execute trajectories efficiently. 

\begin{figure}
\centering
\includegraphics[width=0.9\textwidth]{figs/planninglayers}
\caption{Path-planning across different length scales, requiring a variety of map representations and planning paradigms. Arrows indicate information passed between layers.\label{fig:planninglayers}}
\end{figure}

Note that this representation does not include a reasoning level that encodes traffic rules and common sense. While some of these might be encoded using cost-functions, such as maximizing distance from obstacles or insuring smooth riding, other more complex behaviors such as adapting driving in the presence of cyclists or properties of the ground need to be implemented in an additional vertical layer that has access to all planning layers. 

\section{Other path-planning applications}
Once the environment has been discretized into a graph, we can employ other algorithms from graph theory to plan desirable robot trajectories. For example, floor coverage can be achieved by performing a depth-first search (DFS) or a breadth-first-search (BFS) on a graph where each vertex has the size of the coverage tool of the robot. ``Coverage'' is not only interesting for cleaning a floor: the same algorithms can be used to perform an exhaustive search of a configuration space, such as in the example shown in Figure \ref{fig:inversekinematics}, where we plotted the error of a manipulator arm in reaching a desired position over its configuration space. Finding a minimum in this plot using an exhaustive search solves the inverse kinematics problem. Similarly, the same algorithm can be used to systematically follow all links on a website till a desired depth (or actually retrieving the entire world-wide web).

Doing a DFS or a BFS might generate efficient coverage paths, but they are far from optimal as many vertices might be visited twice. A path that connects all vertices in a graph but passes every vertex only once is known as a \emph{Hamiltonian Path}.\index{Hamiltonian Path} A Hamiltonian path that returns to its starting vertex is known as a Hamiltonian Cycle. This problem is also known as the Traveling Salesman Problem (TSP), in which a route needs to be calculated that visits every city on his tour only once and is known to be NP Complete.\index{Traveling Salesman Problem}



\section{Summary and Outlook}
Path planning is an ongoing research problem. Finding collision free paths for mechanisms with high degrees of freedom such as multiple arms operating in a common space, multi-robot systems, or systems involving dynamics (and therefore adding the derivatives of the state variables to the planning problem) might take unacceptably long to solve.

Although sampling-based path planners can drastically speed up the time to find some solution, there are not optimal and struggle with specific situations such as narrow passages. There is no ``silver bullet'' algorithm for solving all path planning problems and heuristics that lead to massive speed-up in one scenario might be detrimental in others. Also, algorithmic parameters are mostly ad-hoc and correctly tuning them to a specific environment might drastically increase performance.
 

\section*{Take-home lessons}
\begin{itemize}
\item The first step in path planning is choosing a map representation that is appropriate to the application.
\item The second step is to reduce the robot to a point-mass, which allows planning in the configuration space.
\item This allows the application of generic shortest path finding algorithms, which have applications in a large variety of domains, not limited to robotics.
\item A sampling-based planning algorithm finds paths by sampling random points in the environment. Heuristics are used to maximize the exploration of space and bias the direction of search. This makes these algorithms fast, but neither optimal nor complete.
\item As the resulting paths are random, multiple trials might lead to totally different results.
\item There is no one-size-fits-all algorithm for a path-planning algorithm and care must be taken to select the right paradigm (single-query vs. multi-query), heuristics, and parameters.
\end{itemize}

\section*{Exercises}\small
\begin{enumerate}
\item How does the computational complexity of Dijkstra's algorithm change when moving from 2D to 3D search spaces?
\item A* uses a ``heuristic'' to bias the search in the expected direction of the goal. Why can it only use a heuristic, not the actual length?
\item Assuming points are sampled uniformly at random in a randomized planning algorithm. Calculate the limiting behaviour of the following ratio   (number of points in tree)/(number of points sampled) as the number of points sampled goes to infinity. Assume the total area $A_{total}$ and the area of free space $A_{free}$ within are known. 

\item Assuming a kd-tree is used as a nearest-neighbour data structure, and points are sample uniformly at random, calculate the  run-time of inserting a point into a tree of size $N$. Use ``big-Oh'' notation, e.g. $\mathcal{O}(N)$.

\item What other practical runtime concerns must one consider besides computational complexity alone when doing sampling based motion planning? Can you suggest ways to deal with these other concerns?
\end{enumerate}

\normalsize
\input{chapters/sensors}
\input{chapters/vision}
\input{chapters/features}
\input{chapters/errorpropagation}
\input{chapters/localization}

\input{chapters/grasping}



\chapter{Simultaneous Localization and Mapping}\label{chap:slam}
Robots are able to keep track of their position using a model of the noise arising in their drive train and their forward kinematics to propagate this error into a spatial probability density function (Section \ref{sec:errorprop}). The variance of this distribution can shrink as soon as the robot sees uniquely identifiable features with known locations. This can be done for discrete locations using Bayes' rule (Section \ref{sec:markovloc}) and for continuous distributions using the Extended Kalman Filter (Section \ref{sec:ekf}). The key insight here was that every observation will reduce the variance of the robot's position estimate. Here, the Kalman filter performs an optimal fusion of two observations by weighting them with their variance, i.e., unreliable information counts less than reliable one. In the robot localization problem, one of the observations is typically the robot's position estimate whereas the other observation comes from a feature with known location on a map. So far, we have assumed that these locations are known. This chapter will introduce

\begin{itemize}
\item the concept of covariance (or, what all the non-diagonal elements in the covariance matrix are about),
\item how to estimate the robot's location and that of features in the map at the same time (Simultaneous Localization and Mapping or SLAM)
\end{itemize}

\section{Introduction}
The SLAM problem has been considered as the holy grail of mobile robotics for a long time. This lecture will introduce one of the first comprehensive solutions to the problem, which has now be superseded by computationally more efficient versions. We will begin with studying a series of special cases.

\subsection{Special Case I: Single Feature}
Consider a map that has only a single feature. We assume that the robot is able to obtain the relative range and angle of this feature, each with a certain variance. An example of this and how to calculate the variance of an observation based on sensor uncertainty is described in the line fitting example (Section \ref{sec:linefitting}). This feature could be a wall, but also a graphical tag that the robot can uniquely identify. The position of this measurement $m_i=[\alpha_i,r_i]$  in global coordinates is unknown, but can now easily be calculated if an estimate of the robot's position $\boldsymbol{\hat{x}_k}$ is known.  The variance of $ m_i$'s components is now the variance of the robot's position plus the variance of the observation.

Now consider the robot moving closer to the obstacle and obtaining additional observations. Although its uncertainty in position is growing, it can now rely on the feature $m_i$ to reduce the variance of its old position (as long as its known that the feature is not moving). Also, repeated observations of the same feature from different angles might improve the quality of its observation. The robot has therefore a chance to keep its variance very close to that with which it initially observed the feature and stored it into its map. We can actually do this using the EKF framework from Section \ref{sec:EKF}. There, we assumed that features have a known location (no variance), but that the robot's sensing introduces a variance. This variance was propagated into the covariance matrix of the innovation ($ \boldsymbol{S}$). We can now simply add the variance of the estimate of the feature's position to that of the robot's sensing process.

\subsection{Special Case II: Two Features}
Consider now a map that has two features. Visiting one after the other, the robot will be able to store both of them in its map, although with a higher variance for the feature observed last. Although the observations of both features are independent from each other, the relationship between their variances depend on the trajectory of the robot. The differences between these two variances are much lower if the robot connect them in a straight line than when it performs a series of turns between them. In fact, even if the variances of both features are huge (because the robot has already driven for quite a while before first encountering them), but the features are close together, the probability density function over their distance would be very small. The latter can also be understood as the covariance of the two random variables (each consisting of range and angle). In probability theory, the covariance is the measure of how much two variables are changing together. Obviously, the covariance between the locations of two features that are visited immediately after each other by a robot is much higher as those far apart. It should therefore be possible to use the covariance between features to correct estimates of features in retrospect. For example, if the robot returns to the first feature it has observed, it will be able to reduce the variance of its position estimate. As it knows that it has not traveled very far since it observed the last feature, it can then correct this feature's position estimate.

\section{The Covariance Matrix}
When estimating quantities with multiple variables, such as the position of a robot that consists of its x-position, its y-position and its orientation, matrix notation has been a convenient way of writing down equations. For error propagation, we have written the variances of each input variable into the diagonal of a covariance matrix. For example, when using a differential wheel robot, uncertainty in position expressed by $ \sigma_x, \sigma_y$ and $ \sigma_{\theta}$ were grounded in the uncertainty of its left and right wheel. We have entered the variances of the left and right wheel into a 2x2 matrix and obtained a 3x3 matrix that had $ \sigma_x, \sigma_y$ and $ \sigma_{\theta}$ in its diagonal. Here, we set all other entries of the matrix to zero and ignored entries in the resulting matrix that were not in its diagonal. The reason we could actually do this is because uncertainty in the left and right wheel are independent random processes: there is no reason that the left wheel slips, just because the right wheel slips.  Thus the covariance --- the measure on how much two random variables are changing together --- of these is zero. This is not the case for the robot's position: uncertainty in one wheel will affect all output random variables ($ \sigma_x, \sigma_y$ and $ \sigma_{\theta}$) at the same time, which is expressed by their non-zero covariances --- the non-zero entries off the diagonal of the output covariance matrix.

\section{EKF SLAM}\label{sec:ekfslam}\label{sec:ekf}
The key idea in EKF SLAM is to extend the state vector from the robot's position to contain the position of all features. Thus, the state
\begin{equation}
\hat{\boldsymbol{x}}_{k'|k-1}=(x,y,\theta)^T
\end{equation}
becomes
\begin{equation}
\hat{\boldsymbol{x}}_{k}=(x,y,\theta,\alpha_1,r_1,\ldots,\alpha_N,r_N)^T
\end{equation}
assuming $ N$ features, which is a $(3+2N) x1$ vector. The action update (or ``prediction update") is identical to that if features are already known; the robot simply updates its position using odometry and updates the variance of it s position using error propagation. The covariance matrix is now a $(3+2N)x(3+2N)$ matrix that initially holds the variances on position and those of each feature in its diagonal.

The interesting things happen during the perception update. Here it is important that only one feature is observed at a time. Thus, if the robot observes multiple features at once, one needs to do multiple, consecutive perception updates. Care needs to be taken that the matrix multiplications work out. In practice you will need to set only those values of the observation vector (a $(3+2N)x1$ vector) that correspond to the feature that you observe. Similar considerations apply to the observation function and its Jacobian.

\section{Graph-based SLAM}
Usually, a robot obtains an initial estimate of where it is using some onboard sensors (odometry, optical flow, etc.) and uses this estimate to localize features (walls, corners, graphical patterns) in the environment. As soon as a robot revisits the same feature twice, it can update the estimate on its location. This is because the variance of an estimate based on two independent measurements will always be smaller than any of the variances of the individual measurements. As consecutive observations are not independent, but rather closely correlated, the refined estimate can then be propagated along the robot's path. This is formalized in EKF-based SLAM. A more intuitive understanding is provided by a spring-mass analogy: each possible pose (mass) is constrained to its neighboring pose by a spring. The higher the uncertainty of the relative transformation between two poses (e.g., obtained using odometry), the weaker the spring. Every time a robot gains confidence on a relative pose, the spring is stiffened instead. Eventually, all poses will be pulled in place. This approach is known as \emph{Graph-based SLAM}\index{Graph-based SLAM}. 

\subsection{SLAM as a Maximum-Likelihood Estimation Problem}
The classical formulation of SLAM describes the problem as maximizing the posterior probability of all points on the robot's trajectory given the odometry input and the observations. Formally,
\begin{equation}
p(x_{1:T},m|z_{1:T},u_{1:T})
\end{equation}
where $ x_{1:T}$ are all discrete positions from time 1 to time $ T$, $ z$ are the observations, and $ u$ are the odometry measurements. This formulation makes heavily use of the temporal structure of the problem. In practice, solving the SLAM problem requires
\begin{enumerate}
\item A motion update model, i.e., the probability $ p(x_t|x_{t-1},u_t)$ to be at location $ x_t$ given an odometry measurement $ u_t$ and being at location $ x_{t-1}$.
\item  A sensor model, i.e., the probability $ p(z_t|x_t,m_t)$ to make observation $ z_t$ given the robot is at location $ x_t$ and the map $ m_t$.
\end{enumerate}
A possible solution to this problem is provided by the Extended Kalman Filter, which maintains a probability density function for the robot pose as well as the positions of all features on the map. Being able to uniquely identify features in the environment is of outmost importance and is known as the data association problem. Like EKF-based SLAM, graph-based SLAM does not solve this problem and will fail if features are confused.

In graph-based SLAM, a robot's trajectory forms the nodes of a graph whose edges are transformations (translation and rotation) that have a variance associated with it. An alternative view is the spring-mass analogy mentioned above. Instead of having each spring wiggle a node into place, graph-based SLAM aims at finding those locations that maximize the joint likelihood of all observations. As such, graph-based SLAM is a \emph{maximum likelihood estimation}\index{Maximum Likelihood Estimation} problem.

Lets revisit the normal distribution:
\begin{equation}
\frac{1}{\sigma\sqrt{2\pi}}e^{\frac{-(x-\mu)^2}{2\sigma^2}}
\end{equation}

It provides the probability for a measurement to have value $ x$ given that this measurement is normal distributed with mean $ \mu$ and variance $ \sigma^2$.  We can now associate such a distribution with every node-to-node transformation, aka constraint. This can be pairs of distance and angle, e.g. In the literature the measurement of a transformation between node i and a node j is denoted $ z_{ij}$. Its expected value is denoted $ \hat{z}_{ij}$. This value is expected for example based on a map of the environment that consists of previous observations.

Formulating a normal distribution of measurements $ z_{ij}$ with mean $ \hat{z}_{ij}$ and a covariance matrix $ \Sigma_{ij}$ (containing all variances of the components of $ z_{ij}$ in its diagonal) is now straightforward. As graph-based SLAM is most often formulated as  information filter, usually the inverse of the covariance matrix (aka information matrix) is used, which we denote by $ \Omega_{ij}=\Sigma_{ij}^{-1}$.

As we are interested in maximizing the joint probability of all measurements $ \prod{z_{ij}}$ over all edge pairings $ ij$ following the maximum likelihood estimation framework, it is customary to express the PDF using the log-likelihood. By taking the natural logarithm on both sides of the PDF expression, the exponential function vanishes and $ ln \prod{z_{ij}}$ becomes $ \sum{ ln z_{ij}}$ or $ \sum{l_{ij}}$, where $ l_{ij}$ is the log-likelihood distribution for $ z_{ij}$.
\begin{equation}
l_{ij} \propto (z_{ij}-\hat{z}_{ij}(x_i,x_j))^T\Omega_{ij}(z_{ij}-\hat{z}_{ij}(x_i,x_j))
\end{equation}

Again, the log-likelihood for observation $ z_{ij}$ is directly derived from the definition of the normal distribution, but using the information matrix instead of the covariance matrix and is ridden of the exponential function by taking the logarithm on both sides.

The optimization problem can now be formulated as
\begin{equation}
x^* = \arg \min_{x}\sum_{<i,j>\in \mathcal{C}}e_{ij}^T\Omega_{ij}e_{ij}
\end{equation}
with $ e_ij(x_i,x_j)=z_{ij}-\hat{z}_{ij}(x_i,xj)$ the error between measurement and expected value. Note that the sum actually needs to be minimized as the individual terms are technically the negative log-likelihood.

\subsection{Numerical Techniques for Graph-based SLAM}
Solving the MLE problem is non-trivial, especially if the number of constraints, i.e., observations that relate one feature to another, provided is large. A classical approach is to linearize the problem at the current configuration and reducing it to a problem of the form $ Ax=b$. The intuition here is to calculate the impact of small changes in the positions of all nodes on all $ e_{ij}$. After performing this motion, linearization and optimization can be  repeated until convergence. 

More recently, more powerful numerical methods have been developed. Instead of solving the MLE, on can employ a stochastic gradient descent algorithm. A gradient descent algorithm is an iterative approach to find the optimum of a function by moving along its gradient. Whereas a gradient descent algorithm would calculate the gradient on a fitness landscape from all available constraints, a stochastic gradient descent picks only a (non-necessarily random) subset. Intuitive examples are fitting a line to a set of $n$ points, but taking only a subset of these points when calculating the next best guess. As gradient descent works iteratively, the hope is that the algorithm takes a large part of the constraints into account. For solving Graph-based SLAM, a stochastic gradient descent algorithm would not take into account all constraints available to the robot, but  iteratively work on one constraint after the other. Here, constraints are observations on the mutual pose of nodes $i$ and $j$. Optimizing these constraints now requires moving both nodes $i$ and $j$ so that the error between where the robot thinks the nodes should be and what it actually sees gets reduced.  As this is a trade-off between multiple, maybe conflicting observations, the result will approximate a Maximum Likelihood estimate.

More specifically, with $ e_ij$ the error between an observation and what the robot expects to see, based on its previous observation and sensor model, one can distribute the error along the entire trajectory between both features that are involved in the constraint. That is, if the constraint involves features $i$ and $j$, not only $i$ and $j$'s pose will be updated but all points inbetween will be moved a tiny bit.

%This approach is cumbersome and quickly gets out of control if a robot is mapping an environment over multiple hours --- leading to millions of nodes in the graph and constraints. To overcome this problem, [Gris07] propose to (1) merge nodes of a graph as it is build up by relying on accurate localization of the robot within the existing map and (2) to chose a different graph representation.

In Graph-based SLAM, edges encode the relative translation and rotation from one node to the other. Thus, altering a relationship between two nodes will automatically propagate to all nodes in the network. This is because the graph is essentially a chain of nodes whose edges consist of odometry measurements. This chain then becomes a graph whenever observations (using any sensor) introduce additional constraints. Whenever such a ``loop-closure'' occurs, the resulting error will be distributed over the entire trajectory that connects the two nodes. This is not always necessary, for example when considering the robot driving a figure-8 pattern. If a loop-closure occurs in one half of the 8, the nodes in the other half of the 8 are probably not involved.

This can be addressed by constructing a  minimum spanning-tree  (MST) of the constraint graph. The MST is constructed by doing a Depth-First Search (DFS) on the constraint graph following odometry constraints. At a loop-closure, i.e., an edge in the graph that imposes a constraint to a previously seen pose, the DFS backtracks to this node and continues from there to construct the spanning tree. Updating all poses affected by this new constraint still requires modifying all nodes along the path between the two features that are involved, but inserting additional constraints is greatly simplified. Whenever a robot observes new relationships between any two nodes, only the nodes on the shortest path between  the two features on the MST need to be updated. %Example graphs illustrating this are shown in Figures 2a and 2b in [Gris07].
\subsection*{Further reading}
\begin{itemize}
\item G. Grisetti, R. Kuemmerle, C. Stachniss and W. Burgard. A Tutorial on Graph-Based SLAM. IEEE Intelligent Transportation Systems Magazine, 2(4):31-43, 2010.

\item E. Olson, J. Leonard and S. Teller. Fast Iterative Alignment of Pose Graphs with Poor Initial Estimates. Proc. of ICRA, pp 2262-2269, Orlando, FL, 2006.

\item G. Grisetti, C. Stachniss, S. Grzonka and W. Burgard. A Tree Parameterization for Efficiently Computing Maximum Likelihood Maps using Gradient Descent. Robotics: Science and Systems (RSS), Atlanta, GA, USA, 2007.
\end{itemize}

\chapter{RGB-D SLAM} 
 Range sensors have emerged as one of the most effective sensors to make robots autonomous. Unlike vision, range data makes the construction of a 3D model of the robot's environment straightforward and the Velodyne sensor, that combines 64 scanning lasers into one package, was key in mastering the DARPA Grand Challenge.  3D range data has become even more important in robotics with the advent of cheap (priced at a tenth than the cheapest 2D laser scanner) RGB-D (color image plus depth) cameras. Point cloud data allows fitting of lines using RANSAC, which can serve as features in EKF-based localization, but can also be used for improving odometry, loop-closure detection, and mapping. The goals of this chapter are
\begin{itemize}
\item introduce the Iterative Closest Point (ICP) algorithm
\item show how ICP can be improved by providing initial guesses via RANSAC
\item show how SIFT features can be used to improve point selection and loop-closure in ICP to achieve RGB-D mapping
\end{itemize}

\section{Converting range data into point cloud data}
Point cloud data can be thought of a 3D matrix that maps a certain volume in 3D space. Each cell in this matrix, also known as \emph{Voxel}\index{Voxel}, corresponds to whether there is an obstacle in this volume or not. Different intensity values could correspond to the uncertainty with which this space is to be known to be an obstacle. An efficient method to turn range information into such an uncertainty 3D map is described in \cite{curless96} and became known as \emph{Truncated Surface Distance Function} (TSDF)\index{Truncated Surface Distance Function (TSDF)}\index{TSDF}, commonly referred to as ``Point cloud''\index{Point cloud}.

\section{The Iterative Closest Point (ICP) algorithm}
The \emph{Iterative Closest Point} (ICP)\index{Iterative Closest Point}\index{ICP} algorithm was presented in the early 1990ies for registration of 3D range data to CAD models of objects. A more in-depth overview of what is described here is given in \cite{rusinkiewicz01}. The key problem can be reduced to find the best transformation that minimizes the distance between two point clouds. This is the case when matching snapshots from a range sensor or matching a range image with a point cloud sampled from a 3D representation of an object.

In robotics, ICP found an application to match scans from 2D laser range scanners. For example, the transformation that minimizes the error between two consecutive snapshots of the environment is proportional to the motion of the robot. This is a hard problem as it is unclear, which points in the two consecutive snapshots are ``pairs", which of the points are outliers (due to noisy sensors), and which points need to be discarded as not all points overlap in both snapshots. Stitching a series of snapshots together theoretically allows to create a 2D map of the environment. This is difficult, however, as the error between every snapshots ---similar to odometry --- accumulates.   The ICP algorithm also works in 3D where it allows to infer the change in 6D pose of a camera and creation of 3D maps. In addition, ICP has proven useful for identifying objects from a database of 3D objects.

Before providing a solution to the mapping problem, we will focus on the ICP algorithm to match 2 consecutive frames. Variants of the ICP algorithm can be broken down into 6 consecutive steps:
\begin{enumerate}
\item Selection of points in one or both meshes or point clouds.
\item Matching/Pairing these points to samples in the other point cloud/mesh.
\item Weighting the corresponding pairs.
\item Rejecting certain pairs.
\item Assigning an error metric based on the point pairs.
\item Minimizing the error metric.
\item Point Selection
\end{enumerate}
Depending on the number of points generated by the range sensor, it might make sense to use only a few selected points to calculate the optimal transformation between two point clouds, and then test this transformation on all points. Depending on the source of the data, it also turns out that some points are more suitable than others as it is easier to identify matches for them. This is the case for RGB-D data, where SIFT features have been used successfully. This is also the case for planar objects with grooves, where sampling should ensure that angles of normal vectors of sampling points are broadly distributed. Which method to use is therefore strongly dependent on the kind of data being used and should be considered for each specific problem.

\subsection{Matching Points}
The key step in ICP is to match one point to its corresponding point.  For example, a laser scanner hits a certain point at a wall with its 67th ray. After the scanner has been moved by 10cm , the closest hit on the wall to this point might have been by the 3rth ray of the laser. Here, it is actually very unlikely that the laser hits the exact same point on the wall twice, therefore introducing a non-zero error even for optimal pairing. Prominent methods are to find the closest point in the other point cloud or to find the intersection of the source points normal with the destination surface (for matching point clouds to meshes). More recently, SIFT has allowed to match points based on their visual appearance. Similarly to sorting through SIFT features, finding the closest matching point can be accelerated by representing the point cloud in a k-d tree.

\subsection{Weighting of Pairs}
As some pairs are better matches than others, weighting them in some smart way might drastically improve the quality of the resulting transformation. One approach is to give more weight to points that have smaller distances from each other. Another approach is to take into account the color of the point (in RGB-D images) or use the distance of their SIFT features (weighting pairs with low distances higher than pairs with high distances). Finally, expected noise can be used to weight pairings. For example, the estimates made by a laser scanner are much more faithful when taken orthogonally to a plane than when taken at a steep angle.

\subsection{Rejecting of Pairs}
A key problem in ICP are outliers either from sensor noise or simply from incomplete overlap between two consecutive range images. A prime approach in dealing with this problem is to reject pairings of which one of the points lies on a boundary of the point cloud as these points are likely to match with points in non-overlapping regions. As a function of the underlying data, it might also make sense to reject pairings with too high of a distance. This is a threshold-based equivalent to distance-based weighting as described above.

\subsection{Error Metric and Minimization Algorithm}
After points have been selected and matched, pairs have been weighted and rejected, the match between two point clouds needs to be expressed by a suitable error metric, which needs then to be minimized. A straightforward approach is to consider the sum of squared distances between each pair. This formulation can often be solved analytically. Let
\begin{eqnarray}
A=\{a_1,\ldots,a_m\}\\
B=\{b_1,\dots,b_n\}
\end{eqnarray}
be point clouds in $ \mathbb{R}^n$. The goal is now to find a vector $ t \in \mathbb{R}^n$ so that an error function $ \phi(A+t,B)$ is minimized. In 6D (translation and rotation), an equivalent notation can be found for a transformation (see forward kinematics). An error function for the squared distance is then given by
\begin{equation}
\phi(A+t,B)=\frac{1}{m}\sum_{a \in A}\|a+t-N_B(a+t)\|^2
\end{equation}
Here $ N_B(a+t)$ is a function that provides the nearest neighbor of $ a$ translated by $ b$ in $ B$.  A key problem now is that the actual value of $t$ affects the outcome of the pairing. What might look like a good match initially often turns out not be the final pairing. A simple numerical approach to this problem is to find $ t$ iteratively.

Initially $t=0$ and nearest neighbors/pairings are established. We can now calculate a $ \delta t$ that optimizes the least-square problem based on this matching using any solver available for the optimization problem (for a least-square solution $ \delta t$ can be obtained analytically by solving for the minimum of the polynomial by setting its derivative to zero). We can then shift all points in $ A$ by $ \delta t$ and start over. That is, we calculate new pairings and derive a new $ \delta t$.  We can continue to do this, until the cost function reaches a local minimum.

Instead of formulating the cost function as a ``point-to-point'' distance, a ``point-to-plane'' has become popular. Here, the cost function consist of the sum of squared distances from each source point to the plane that contains the destination point and is oriented perpendicular to the destination normal. This makes particularly sense when matching a point cloud to a mesh/CAD model of an object. In this case there are no analytical solutions to finding the optimal transformation, but any optimization method such a Levenberg-Marquardt can be used.

\section{RGB-D Mapping}
The ICP algorithm can be used to stitch consecutive range images together to create a 3D map of the environment \cite{henry2010rgb}. Together with RGB information, it is possible to create complete 3D walk throughs of an environment. An example of such a walk through using the method described in \cite{whelan2013robust} is shown in Figure \ref{fig:kintinous}.
A problem with ICP is that errors in each transformation propagate making maps created using this method as odd as maps created by simple odometry. Here, the SLAM algorithm can be used to correct previous errors once a loop closure is detected.


\begin{figure}
\centering
\includegraphics[width=\textwidth]{figs/kintinous}
\caption{Fused point cloud data from a walk trough of an office environment using ``Kintinious''. Picture courtesy of John Leonard.\label{fig:kintinous}}
\end{figure}

The intuition behind SLAM is to consider each transformation between consecutive snapshots as a spring with variable stiffness. Whenever the robot returns to a previously seen location, i.e., a loop-closure has been determined, additional constraints are introduced and the collection of snapshots connected by springs become a mesh. Everytime the robot then re-observes a transformation between any of the snapshots, it can ``stiffen'' the spring connecting the two. As all of the snapshots are connected, this new constraints propagates through the network and literally pull each snapshots in place.

RGB-D Mapping uses a variant of ICP that is enhanced by SIFT features for point selection and matching. Maps are build incrementally. SIFT features, and their spatial relationship, are used for detecting loop closures. Once a loop closure is detected, an additional constraint is added to the pose graph and a SLAM-like optimization algorithm corrects the pose of all previous observations.

As ICP only works when both point clouds are already closely aligned, which might not be the case for a fast moving robot with a relatively noisy sensor (the XBox Kinect has an error of 3cm for a few meters of range vs. millimeters in laser range scanners), RGB-D Mapping uses RANSAC to find an initial transformation. Here, RANSAC works as for line fitting: it keeps guessing possible transformations for 3 pairs of SIFT feature points and then counts the number of inliers when matching the two point clouds, one of which being transformed using the random guess.



\appendix
\chapter{Trigonometry}
Trigonometry relates angles and lengths of triangles. Figure \ref{fig:triangle} shows a right-angled triangle and conventions to label its corners, sides, and angles. In the following, we assume all triangles to have at least one right angle (90 degrees or $\frac{\pi}{2})$ as all planar triangles can be dissected into two right-angled triangles. 

\begin{figure}[!htb]
\centering
\includegraphics[width=0.8\textwidth]{figs/triangle}
\caption{Left: A right-angled triangle with common notation. Right: Trigonometric relationships on the unit circle and angles corresponding to the four quadrants. \label{fig:triangle}}
\end{figure}

The sum of all angles in any triangle is 180 degrees or $2\pi$, or 
\begin{equation}
\alpha + \beta + \gamma = 180^o
\end{equation}
If the triangle is right-angled, the relationship between edges $a$, $b$, and $c$, where $c$ is the edge opposite of the right angle is
\begin{equation}
a^2+b^2=c^2
\end{equation} 
The relationship between angles and edge lengths are captured by the trigonometric functions:
\begin{eqnarray}
\sin{\alpha}&=\frac{opposite}{hypothenuse}=\frac{a}{c}\\
\cos{\alpha}&=\frac{adjacent}{hypothenuse}=\frac{b}{c}\\
\tan{\alpha}&=\frac{opposite}{adjacent}=\frac{\sin{\alpha}}{\cos{\alpha}}=\frac{a}{b}
\end{eqnarray} 

Here, the \emph{hypothenuse}\index{Hypothenuse} is the side of the triangle that is opposite to the right angle. The \emph{adjacent} and \emph{opposite} are relative to a specific angle. For example, in Figure \ref{fig:triangle}, the adjacent of angle $\alpha$ is side $b$ and the opposite of $\alpha$ is edge $a$. 

Relations between a single angle and the edge lengths are captured by the \emph{law of cosines}\index{Law of Cosines}:
\begin{equation}
a^2=b^2+c^2-2bc\cos{\alpha}
\end{equation}

\section{Inverse trigonometry}
In order to calculate an angle given two edges, one uses inverse functions $\sin^{-1}$, $\cos^{-1}$, and $\tan^{-1}$. (Not to be confused with $\frac{1}{\sin}$ etc.) As functions can, by definition, only map one value to exactly one other value, $\sin^{-1}$ and $\tan^{-1}$ are only defined in the interval $[-90^o;+90^o]$ and $\cos^{-1}$ is defined in the interval $[0^o;180^o]$. This makes it impossible to calculate angles in the 2nd and 3rd, or the 3rd and 4th quadrant, respectively (Figure \ref{fig:triangle}). 
In order to overcome this problem, most programming languages implement a function \texttt{atan2(opposite,adjacent)}, which evaluates the sign of the numerator and denumerator, provided as two separate parameters. 

\section{Trigonometric identities}
Sine and cosine are periodic, leading to the following identities:
\begin{eqnarray}
\sin\theta=-\sin(-\theta)=-\cos(\theta+\frac{\pi}{2})=\cos(\theta-\frac{\pi}{2})\\
\cos\theta=\cos(-\theta)=\sin(\theta+\frac{\pi}{2})=-\sin(\theta-\frac{\pi}{2})
\end{eqnarray}

The sine or cosine for sums or differences between angles can be calculated using the following identities:

\begin{eqnarray}
\cos(\theta_1+\theta_2)=c_{12}=c_1c_2-s_1s2\\
\sin(\theta_1+\theta_2)=s_{12}=c_1s_2+s_1c_2\\
\cos(\theta_1-\theta_2)=c_1c_2+s_1s_2\\
\sin(\theta_1-\theta_2)=s_1c_2-c_1s_2
\end{eqnarray}

The sum of the squares of sine and cosine for the same angle is one:
\begin{equation}
\cos(\theta)\cos(\theta)+\sin(\theta)\sin(\theta)=1
\end{equation}

\chapter{Linear Algebra}
Linear algebra concerns vector spaces and linear mappings between them. It is central to robotics as it allows describing positions and speeds of the robot within the world as well as moving parts connected to it. 

\section{Dot product}
The dot product (or scalar product)\index{Dot product}\index{Scalar product} is the sum of the products of the individual entries of two vectors. Let $\hat{a}=(a_1,\ldots,a_i)^T$ and $\hat{b}=(b_1,\ldots,b_i)$ be two vectors. Then, there dot product $\hat{a}\cdot\hat{b}$ is given by
\begin{equation}
\hat{a}\cdot\hat{b}=\sum_{i}=a_ib_i
\end{equation}
The dot product therefore takes two sequences of numbers and returns a single scalar. 

In robotics, the dot product is mostly relevant due to its geometric interpretation:
\begin{equation}
\hat{a}\cdot\hat{b}=\|\hat{a}\|\|\hat{b}\|\cos\theta
\end{equation}
with $\theta$ the angle between vectors $\hat{a}$ and $\hat{b}$. 

If $\hat{a}$ and $\hat{b}$ are orthogonal, it follows $\hat{a}\cdot\hat{b}=0$. If $\hat{a}$ and $\hat{b}$ are parallel, it follows $\hat{a}\cdot\hat{b}=\|\hat{a}\|\|\hat{b}\|$.  

\section{Cross product}
The cross product $\hat{a} \times \hat{b}$ of two vectors is defined as a vector $\hat{c}$ that is perpendicular to both $\hat{a}$ and $\hat{b}$. Is direction is given by the right-hand rule and its magnitude is equal to the area of the parallelogram that the vectors span.  

Let Let $\hat{a}=(a_1,a_2,a_3)^T$ and $\hat{b}=(b_1,a_2,a_3)$ be two vectors in $\mathrm{R}^3$. Then, there cross product $\hat{a}\times\hat{b}$ is given by
\begin{equation}
\hat{a}\times\hat{b}=\left(
\begin{array}{l}
a_2b_3-a3b2\\
a_3b_1-a_1b_3\\
a_1b_2-a_2b_1
\end{array}
\right)
\end{equation}


\section{Matrix product}
Given an $n \times m$ matrix $\mathbf{A}$ and a $m\times p$ matrix $\mathbf{B}$, the matrix product $\mathbf{AB}$ is defined by
\begin{equation}
(\mathbf{AB})_{ij}=\sum_{k=1}^mA_{ik}B_{kj}
\end{equation}
where the index $ij$ indicates the i-th row and j-th column entry of the resulting $n\times p $ matrix. Each entry therefore consists of the scalar product of the i-th row of $\mathbf{A}$ with the j-th column of $\mathbf{B}$. 

Note that for this two work, the right hand matrix (here $\mathbf{B}$) has to have as many columns as the left hand matrix (here $\mathbf{A}$) has rows. Therefore, the operation is not commutative, i.e., $\mathbf{AB}\neq\mathbf{BA}$.

For example, multiplying a 3x3 matrix with a 3x1 matrix (a vector), works as follows:
Let
\begin{equation}\nonumber
\mathbf{A} = \begin{pmatrix}
a & b & c \\
p & q & r \\
u & v & w
\end{pmatrix} \qquad \mathbf{B} = \begin{pmatrix}
x \\
y \\
z
\end{pmatrix}.
\end{equation}

Then their matrix product is:
\begin{equation}\nonumber
\mathbf{AB} = \begin{pmatrix}
a & b & c \\
p & q & r \\
u & v & w
\end{pmatrix} \begin{pmatrix}
x \\
y \\
z
\end{pmatrix} =\begin{pmatrix}
ax + by + cz \\
px + qy + rz \\
ux + vy + wz
\end{pmatrix}
\end{equation}

\section{Matrix inversion}
Given a matrix $\mathbf{A}$, finding the inverse $\mathbf{B}=\mathbf{A}^{-1}$ involves solving the system of equations that satisfies
\begin{equation}
\mathbf{AB}=\mathbf{BA}=\mathbf{I}
\end{equation}
With $\mathbf{I}$ the identity matrix\index{Identify matrix}. (The identity matrix is zero everywhere expect at its diagonal entries, which are one.)

In the particular case of orthonormal matrices\index{Orthonormal matrix}, which columns are all orthogonal to each other and of length one, the inverse is equivalent to the transpose, i.e.
\begin{equation}
\mathbf{A}^{-1}=\mathbf{A}^T
\end{equation}

In case a matrix is not quadratic, we can calculate the pseudo-inverse\index{Pseudo-inverse}, which is defined by
\begin{equation}
\mathbf{A}^+=\mathbf{A}^T(\mathbf{AA}^T)^{-1}.
\end{equation}

\input{chapters/statistics}

\chapter{How to write a research paper}
The final deliverable of a robotics class often is a write-up on a ``research'' project, modeled after research done in industry or academia. Roughly, there are three classes of papers:

\begin{enumerate}
\item Original research
\item Tutorial
\item Survey
\end{enumerate}

The goal of this chapter is to provide guidelines on how to think about your project as a research project and how to report on your results as original research.

\section{Original}
Classically, a scientific paper follows the following organization:
\begin{enumerate}
\item Abstract
\item Introduction
\item Materials \& Methods
\item Results
\item Discussion
\item Conclusion
\end{enumerate}

The \emph{abstract} summarizes your paper in a few sentences. What is the problem you want to solve, what is the method you are employing, what are you doing to assess your work, and what is the final outcome.

The \emph{introduction} should describe the problem that you are solving and why it is important. A good guideline to write a good introduction are the Heilmeier questions:

\begin{enumerate}
\item What are you trying to do? Articulate your objectives using absolutely no jargon.
\item How is it done today, and what are the limits of current practice?
\item What's new in your approach and why do you think it will be successful?
\item Who cares?
\item If you're successful, what difference will it make?
\item What are the midterm and final ``exams'' to check for success?
\end{enumerate}

Originally conceived for proposal writing by the head of DARPA, there are additional questions including ``What will it cost?'', ``How long will it take?'', and ``What are the risks and pay-off'', which are left out for the purpose of writing a research paper. In the context of scientific research, the question ``What are you trying to do?'' is best answered in the form of a \emph{hypothesis}, see below. 

The \emph{materials \& matters} section describes all the tools that you used to solve your problem, as well as your original contribution, e.g., an algorithm that you came up with. This section is hardly ever labeled as such, but might consist of a series of individual section describing the robotic platform you are using, the software packages, and flowcharts and descriptions on how your system works. Make sure you motivate your design choices using conclusive language or experimental data. Validating these design choices could be your first results. 

The \emph{results} section contains data or proofs on how to solve the problem you addressed or why it cannot be solved. It is important that your data is conclusive! You have to address concerns that your results are just a lucky coincidence. You therefore need to run multiple experiments and/or formally prove the workings of your system either using language or math, see also Section \ref{sec:stattest}.

The \emph{discussion} should address limitations of your approach, the conclusiveness of its results, and general concerns someone who reads your work might have. Put yourself in the role of an external reviewer who seeks to criticize your work. How could you have sabotaged your own experiment? What are the real hurdles that you still need to overcome for your solution to work in practice? Criticizing your own  work does not weaken it, it makes it stronger! Not only does it become clear where its limitations are, it is also more clear where other people can step in. 

The \emph{conclusion} should summarize the contribution of your paper. It is a good place to outline potential future work for you and others to do. This future work should not be random stuff that you could possibly think about, but come out of your discussion and the remaining challenges that you describe there. Another way to think about is that the ``future work'' section of your conclusion summarizes your discussion.

It is important not to mix the different sections up. For example, your result section should exclusively focus on describing your observations and reporting on data, i.e., facts. Don't conjecture here why things came out as they are. You do this either in your hypothesis --- the whole reason you conduct experiments in the first place --- or in the discussion. Similarly, don't provide additional results in your discussion section.

Try to make the paper as accessible to as many reader styles and attention spans as possible. While this sounds impossible at first, a good way to address this is to think about multiple avenues a reader might take. For example, the reader should get a pretty comprehensive picture on what you do by just reading the abstract, just reading the introduction, or just reading all the figure captions. (Think about other avenues, every one you address makes your paper stronger.) It is often possible to provide this experience by adding short sentences  that quickly recall the main hypothesis of your work. For example, when describing your robotic platform in the materials section, it does not hurt to introduce the section by something like ``In order to show that [the main hypothesis of our work], we selected...''. Similarly, you can try to read through your figure captions if they provide enough information to follow the paper and understand its main results on their own. Its not a problem to be repetitive in a scientific paper, stressing your one-sentence elevator pitch (or hypothesis, see below) throughout the paper is actually a good thing.

\section{Hypothesis: Or, what do we learn from this work?}
Classically, a hypothesis is a proposed explanation for an observed phenomenon. From this, the hypothesis has emerged as the corner stone of the scientific method and is a very efficient way to organize your thoughts and come up with a one sentence summary of your work. A proper formulation of your hypothesis should directly lead to the method that you have chosen to test your hypothesis. A good way to think about your hypothesis is ``What do you want to learn?'' or ``What do we learn from this work?''.

It can be somewhat hard to actually frame your work into a single sentence, so what to do if a single hypothesis seems not to apply? One reason might be that you are actually trying to accomplish too many things. Can you really describe them all in depth in a 6-page document? If yes, maybe some are very minor compared to the others.  If this is the case, they are either supportive of your main idea and can be rolled into this bigger piece of work or they are totally disconnected. If they are disconnected, leave them out for the sake of improving the conciseness of your main message. Finally, you might feel that you don't have a main message, but consider all the things you done equally worthy, and despite answering the Heilmeier questions you cannot fill up more than three pages. In this case you might consider picking one of your approaches and dig deeper by comparing it with different methods.

Being able to come up with a one-sentence elevator pitch framed as a hypothesis will actually help you to set the scope of the work that you need to do for a research or class project. How good do you need to implement, design or describe a certain component of your project? Well, good enough to follow through with your research objective.

\section{Survey and Tutorial}
The goal of a \emph{survey} is to provide an overview over a body of work --- potentially from different communities --- and classify it into different categories. Doing this synthesis and establishing common language and formalism is the survey's main contribution.  A survey following such an outline is a possible deliverable for an independent study or a PhD prelim, but it does not lend itself to describe your efforts on a focused research project. Rather, it might result from your involvement in a relatively new area in which you feel important connections between disjoint communities and common language have not been established. 

A different category of survey critically examines concurring methods to solve a particular problem. For example, you might have set out to study manipulation, but got stuck in selecting the right sensor suite from the many available options. What sensor is actually best to accomplish a specific task? A survey which answers this question experimentally will follow the same structure as a research paper (see above).

A \emph{tutorial} is closely related to a survey, but focuses more on explaining specific technical content, e.g, the workings of a specific class of algorithms or tool, commonly used in a community. A tutorial might be an appropriate way to describe your efforts in a research project, which can serve as illustration to explain the workings of a specific method you used.

\section{Writing it up!}
Writing a research report that contains equations, figures and references requires some tedious book-keeping. Although technically possible, word processing programs quickly reach their limitations and will lead to frustration. In the scientific community \LaTeX~ has emerged as a quasi standard for typesetting research documentation. \LaTeX~ is a mark-up language that strictly divides function and layout. Rather than formatting individual items as bold, italic and the like, you mark them up as emphasized, section head etc, and specify how things look elsewhere. This is usually provided by a template provided by the publisher (or your own). While \LaTeX~ has quite a learning curve compared to other word processing software, it is quickly worth the effort as soon as you need to start worrying about references, figures or even indices. 

\section*{Further Reading}

\begin{itemize}
\item W. Strunk and E. White. The Elements of Style (4th Edition). Longan, 1999.
\item T. Oetiker, H. Partl, I. Hyna and E. Schlegl. The Not So Short Introduction to \LaTeXe. Available online.
\end{itemize}

\chapter{Sample curricula}

This book is designed to cover two full semesters at undergraduate level, CSCI 3302 and CSCI 4302 at CU Boulder, or a single semester ``crash course'' at graduate level. There are multiple avenues that an instructor could take, each with their unique theme and a varying set of prerequisites on the students. 
 

\section{An introduction to autonomous \emph{mobile} robots}
This describes a possible one semester curriculum, which takes the students from the kinematics of a differential-wheel platform to SLAM. This curriculum is involved and requires a firm background in trigonometry, probability theory and linear algebra. This might be too ambitious for third-year Computer Science students, but fares well with Aerospace and Electrical Engineering students, who often have a stronger, and more applied, mathematical background. This curriculum is therefore also well suited as ``advanced class'', e.g. in the fourth year of a CS curriculum.

\subsection{Overview}
The curriculum is motivated by a maze-solving competition that is described in Section \ref{sec:ratslife}. Solving the game can be accomplished using a variety of algorithms ranging from wall following (which requires simple proportional control) to Depth-first Search on the maze to full SLAM. Here, the rules are designed such that creating a map of the environment leads to a competitive advantage on the long run.

\subsection{Materials}
The competition can be easily re-created using card board or LEGO bricks and any miniature, differential wheel platform that is equipped with a camera to recognize simple markers in the environment (which serve as landmarks for SLAM). The setup can also easily be simulated in a physics-based simulation environment, which allows scaling this curriculum to a large number of participants. The setup used at CU Boulder using the e-Puck robot and the Webots simulator is shown in Figure \ref{fig:ratslifereal}.

\begin{figure}[!htb]
\includegraphics[width=0.48\textwidth]{figs/ratslife_real}
\includegraphics[width=0.48\textwidth]{figs/ratslife_webots}
\caption{\label{fig:ratslifereal}The ``Ratslife'' maze competition created from LEGO bricks and e-Puck robots (left). The same environment simulated in \emph{Webots}.}
\end{figure}

\subsection{Content}\label{sec:curr1content}
After introducing the field and the curriculum using Chapter \ref{chap:introduction} ``\nameref{chap:introduction}'', another week can be spend on basic concepts from Chapter \ref{chap:locomotion} ``\nameref{chap:locomotion}'', which includes concepts like ``\nameref{sec:stability}'' and ``\nameref{sec:dof}''. The lab portions of the class can at this time be used to introduce the software and hardware used in the competition. For example, students can experiment with the programming environment of the real robot or setup a simple world in the simulator themselves.

The lecture can then take up pace with Chapter \ref{chap:kinematics}. Here, the topics ``\nameref{sec:coordsystems}'', ``\nameref{sec:fwkmobile}'', and ``\nameref{sec:ivkmobile}'' are on the critical path, whereas other sections in Chapter \ref{chap:kinematics} are optional. It is worth mentioning that the forward kinematics of non-holonomic platforms, and in particular the motivation for considering their treatment in velocity rather than position space, are not straightforward and therefore at least some treatment of arm kinematics is recommended. These concepts can easily be turned into practical experience during the lab session. 

The ability to implement point-to-point motions in configuration space thanks to knowledge of inverse kinematics, directly lends itself to ``\nameref{sec:maps}'' and ``\nameref{chap:pathplanning}'' treated in Chapter \ref{chap:pathplanning}. For the purpose of maze solving, simple algorithms like Dijkstra's and A* are sufficient, and sampling-based approaches can be skipped. Implementing a path-planning algorithm both in simulation and on the real robot will provide first-hand experience of uncertainty. 

The lecture can then proceed to ``\nameref{chap:sensors}'' (Chapter \ref{chap:sensors}), which should be used to motivate uncertainty using concepts like accuracy and precision. These concepts can be formalized using materials in Chapter \ref{chap:statistics} ``\nameref{chap:statistics}'', and quantified during lab. Here, having students record the histogram of sensor noise distributions is a valuable exercise. 

Chapters \ref{chap:vision} and \ref{chap:feature_extraction}, which are on ``\nameref{chap:vision}'' and ``\nameref{chap:feature_extraction}'', do not need to extend further than needed to understand and implement simple algorithms for detecting the unique features in the maze environment. In practice, these can usually be detected using basic convolution-based filters from Chapter \ref{chap:vision}, and simple post-processing, introducing the notion of a ``feature'', but without reviewing more complex image feature detectors. The lab portion of the lab should be aimed at identifying markers in the environment, and can be scaffolded as much as necessary. 

Indepth experimentation with sensors, including vision, serves as a foundation for a more formal treatment of uncertainty in Chapter \ref{chap:uncertainty} ``\nameref{chap:uncertainty}''. Depending on whether the ``\nameref{sec:linefitting}'' example has been treated in Chapter \ref{chap:feature_extraction}, it can be used here to demonstrate error propagation from sensor uncertainty, and should be simplified otherwise. In lab, students can actually measure the distribution of robot position over hundreds of individual trials (this is an exercise that can be done collectively if enough hardware is available), and verify their math using these observations. Alternatively, code to perform these experiments can be provided, giving the students more time to catching up. 

The localization problem introduced in Chapter \ref{chap:localization} is best introduced using Markov localization, from which more advanced concepts such as the particle filter and the Kalman filter can be derived. Performing these experiments in the lab is involved, and are best done in simulation, which allows neat ways to visualize the probability distributions changing. 

The lecture can be concluded with ``\nameref{sec:ekfslam}'' in Chapter \ref{chap:slam}. Actually implementing EKF SLAM is beyond the scope of an undergraduate robotics class and is achieved only by very few students who go beyond the call of duty. Instead, students should be able to experience the workings of the algorithm in simulation, e.g., using one of the many available Matlab implementations, or scaffolded in the experimental platform by the instructor. 

The lab portion of the class can be concluded by a competition in which student teams compete against each other. In practice, winning teams differentiate themselves by the most rigorous implementation, often using one of the less complex algorithms, e.g., wall following or simple exploration. Here, it is up to the instructor incentivizing a desired approach. 

Depending on the pace of the class in lecture as well as the time that the instructor wishes to reserve for implementation of the final project, lectures can be offset by debates, as described in Section \ref{sec:debates}. 

\section{An introduction to autonomous manipulation}
Although robotic manipulation is a much less mature field than autonomous mobile robots, teaching its basics, such as those treated in this book, is slightly easier, mainly due to the fact that concepts like uncertainty and non-holonomy are mostly absent. Robotic manipulation is also well suited for a practice-based curriculum due to the wide array of cheap, multi-DOF robotic arms. These, of course, quickly reach their limitations to demonstrate advanced topics such as dynamics or force control, which are beyond the scope of this book.

\subsection{Overview} 
A manipulation-driven curriculum can be motivated by a ``grand challenge'' task such as robotic agriculture, robotic construction or assisted living, all of which have a manipulation problem at their core. Although a class project is likely to be limited to a toy-example, taking advantage of modern motion-planning frameworks and visualization tools, e.g. ROS/Moveit!, makes it easy to put the class into an industry-relevant framework and expose the students to state of the art platforms in simulation.

\subsection{Materials} 
Possible class project range from ``robot gardening'' or ``robots building robots'', for which setups can easily be created. These include real or plastic cherry tomato or strawberry plants and robotic construction kits such as Modular Robotics ``Cubelets'', which easily snap together and have the advantage to form structures that are robots themselves, adding additional motivation. The robot arm, such as the open-source, 7-DOF CLAM arm, can be mounted on a portable structure that contains fixed a set of fixed (3D) cameras. In order to allow a large number of students to get familiar with the necessary software and hardware, the instructor can provide a virtual machine with a preinstalled Linux environment and simulation tools. In particular, using the ``Robot Operating Systems'' (ROS) allows recording so-called ``bag''-files of sensor values, including entire sequences of joint recordings and RGB-D video. This allows the students to work on a large part of the homeworks and project preparation from a computer lab or from home, maximizing availability of real hardware.

\subsection{Content}
The first two weeks of this curriculum can be mostly identical to that described in Section \ref{sec:curr1content}. If a message passing system such as ROS is used, a good exercise is to record a histogram of message passing times in order to get familiar with the software.

In Chapter \ref{chap:kinematics}, the focus is instead on manipulating arms, including the Denavit-Hartenberg scheme and numerical methods for inverse kinematics. In turn, the topics ``\nameref{sec:fwkmobile}'', and ``\nameref{sec:ivkmobile}'' do not necessarily need to be included. Forward and inverse kinematics can be easily turned into lab sessions using Matlab/Mathematica, simulation or a real robot platform. If the class uses a more complex or industrial robot arm, an alternative path is to record joint trajectories in a ROS bag and letting the students explore this data, e.g., sketching 


\section{Class debates}\label{sec:debates}
Class debates are a good way to decompress at the end of class and require the students to put the materials they learned in a broader context. Student teams prepare pro and contra arguments for a statement of current technical or societal concern, exercising presentation and research skills. Sample topics include \emph{Robots putting humans out of work is a risk that needs to be mitigated}; \emph{Robots should not have the capability to autonomously discharge weapons / drive around in cities (autonomous cars)}; or \emph{Robots need to be made from components other than links, joints, and gears in order to reach the agility of people}.

The students are instructed to make as much use as possible of technical arguments that are grounded in the course materials and in additional literature. For example, students can use the inherent uncertainty of sensors to argue for or against enabling robots to use deadly weapons. Similarly, students relate the importance and impact of current developments in robotics to earlier inventions that led to industrialization, when considering the risk of robots putting humans out of work. 

Although suspicious as first, students usually receive this format very well.  While there is agreement that debates help to prepare them for the engineering profession by improving presentation skills, preparing engineers to think about questions posed by society, and reflecting up-to-date topics, the debates seem to have little effect on changing the students' actual opinions on a topic. For example, in a questionnaire administered after class, only two students responded positively. Students are also undecided about whether the debates helped them to better understand the technical content of the class. Yet students find the debate concept important enough that they prefer it over a more in-depth treatment of the technical content of the class, and disagree that debates should be given less time in class. However, students are undecided whether debates are important enough to merit early inclusion in the curriculum or to be part of every class in engineering. 

Concerning the overall format, students find that discussion time was too short when allotting 10 minutes per position and 15 minutes for discussion and rebuttal. Also, students tend to agree that debates are an opportunity to decompress (``relaxing''), which is desirable as this period of class coincides with wrapping up the course project.

\bibliographystyle{agsm}
\bibliography{robotics}

\printindex

\end{document}
